\chap Spectroscopy

Blablabla~\cite[Bennett2005].

\sec Light

The {\bf visible light} is only a tiny part of the complete spectrum of light.
The complete spectrum of light is usually called the {\bf electromagnetic
spectrum}. Light itself is often called {\bf electromagnetic radiation}.

In general, a wave is something that can transmit energy without carrying
material along with it. The {\bf wavelength} is defines as the distance between
adjacent peaks and the {\bf frequency} as the number of times that any piece
of the rope moves up and down each second.

If electrons could be lined up, they would wriggle up and down as light passes
by, demonstrating that light is a wave. Because light can affect both
electrically charged particles and magnets it is called {\em electromagnetic
wave}.

All light travels at the same speed, which is the {\bf speed of light} ($c =
300000\,\rm km\,s^{-1}$). This fact explains that {\em the longer
the wavelength, the lower the frequency, and vice versa}.

Experiments show that light can behave {\em both} as a wave and a particle.
The particle nature of light arises because light comes in pieces called
{\bf photons}. Light can be think of as consisting of many photons
characterized by a wavelength and a frequency. In addition, each photon
has a amount of energy that depend on frequency (the higher frequency the more
energy).

Visible light has wavelengths from 400\,nm at blue or violet end to about
700\,nm at the red end. Light with wavelengths beyond the red end is called
{\bf infrared} and the longest wavelengths have the {\bf radio waves}.

On the other side of the spectrum shorter than blue light is the {\bf
ultraviolet} light, followed by {\bf X rays} and light with the shortest
wavelengths is called {\bf gamma rays}.

\sec Matter

Planets, stars and galaxies are made of matter. Light carries information about
the matter across the universe.

Chemical elements are made from a different types of {\bf atoms}, which are made
of particles ({\bf protons}, {\bf neutrons} and {\bf electrons}). Protons
and neutrons are in {\bf nucleus} at the center of a atom. The electrons
surround the nucleus.

The atom's properties are mainly determined by {\bf electrical charge} in its
nucleus (protons has positive charge, electrons has negative charge
and neutrons are electrically neutral). Electrical charge is fundamental
property which is always conserved just as energy is always conserved.

\sec Light and Matter Interaction

Energy carried by light can interact with matter in these ways:

\begitems
* {\bf emission}: electric current flowing through the bulb heats it to a point
at witch it {\em emits} visible light,
* {\bf absorption}: hand placed near a lit light bulb {\em absorbs} some
of the light,
* {\bf transmission}: some matter allow light to pass through,
* {\bf reflection} or {\bf scattering}: photons can bounce off matter leading to
{\em reflection} or {\em scattering}.
\enditems

\sec Types of Light Spectra

Spectra come in three basic types. Real astronomical spectra are usually
a combination of these types.

\begitems \style n
* The spectrum of a common light bulb is a continuous rainbow. It is called
{\bf thermal radiation spectrum}.
* If a cloud of gas lies between detector and a bulb, the cloud can absorb
specific wavelength making what is called {\bf absorption line spectrum}.
* The cloud might emit light itself therefore this spectrum is called
{\bf emission line spectrum}.
\enditems

Spectra can be displayed as {\em bands of light} (projection of light that
passes through a prism on a wall) or as {\em graphs} (if an intensity
of the light is each wavelength is measured).

\sec Light Tells What Things Are Made Of

Positions of emission and absorption lines can tell what objects are made of.

Electrons can have only particular amount of energy. The lowest possible energy
level is called {\em level 1} or the {\em ground state}. An electron in level 1
can jump to higher level only if it gains precise amount of energy. If
an electron gains enough energy to reach the {\em ionization level}, it escapes
the atom completely.

Other atoms, ions and molecules have their own distinct sets of energy levels.

The fact that each atom, ion or molecule possesses a unique set of energy levels
causes emission and absorption lines at specific wavelengths in spectra.
Electrons absorbs the right amount of energy but they can't stay in higher
energy level for long, so they emits the energy in form of photons and fall
back to lower level. Using this spectral information it is possible to detect
presence of atoms in distant object.

\sec Thermal Radiation

Thermal radiation (sometimes known as {\em black body radiation}) spectrum can
help us determine the temperature of the object producing it.

Photons tend to bound randomly around inside a dense or large object and by
the time they escape they are randomly speared over a wide range of wavelengths.
This explains why a spectrum of light from such object seems {\em continuous}.

The spectrum only depends on object's {\em temperature}. The temperature has
two effects on the spectrum:

\begitems
* Hotter objects show greater intensity at all wavelengths, indicating that
they emit more total radiation per unit area.
* Hotter object emit photons with a higher average energy.
\enditems

\sec Doppler Effect

Doppler effect can tell how fast is an object moving toward or away form
an observer. If an object is moving toward an observer the light waves bunch
up so that the whole spectrum is shifted to shorter wavelengths. This is called
{\bf blueshift}. If an object is moving away from an observer the wavelengths
shift is called {\bf redshift}.

Important is concept of atoms {\bf rest wavelengths}. These are obtained
in laboratory stationary setting. By observing certain pattern of absorption
or emission lines it is possible to determine relative speed of a moving
object. It is unable to tell the direction.

\sec Telescopes

Telescopes are giant eyes that can collect much more light that human's naked
eye. Sophisticated cameras can make high-quality images of light collected
by a telescope and spectrographs can disperse the light into a spectra which
reveal objects composition, speed, temperature and more.

The two fundamental properties of telescope are its {\em light collecting area}
and {\em angular resolution}. {\bf Light collecting area} tells how much light
can the telescope collect at one time usually by stating the {\em diameter} of
light collecting area. {\bf Angular resolution} tells the smallest angle over
which two dots are distinct.

There are two basic designs of telescopes. A {\bf refracting telescope}
operates like a human's eye using transparent glass lenses to focus the light
from distant objects. A {\bf reflecting telescope} uses {\em primary mirror} to
gather light. This mirror reflects the light to a {\em secondary mirror} which
reflects the light to focus where it can be observed. Nearly all telescopes
in research are reflectors.
