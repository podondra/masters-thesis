\chapter{Domain Adaptation (DA)}
\label{da_chapter}

Survey the current state of domain adaptation using neural network models in machine learning and with a focus on astronomical applications.

\section{DA in Context of Machine and Transfer Learning}

Now, we introduce the crucial concept of our work: domain adaptation.
Domain adaptation is a subfield of transfer learning,
which is part of machine learning.
Transfer learning is defined in most papers regarding the survey by Pan and Yang~\cite{pan2010}.
A more recent survey by Weiss, Khoshgoftaar and Wang~\cite{weiss2016} has the benefit
that it contains newer methods than the survey by Pan and Yang~\cite{pan2010},
but its definition of transfer learning and domain adaptation is the same.

Machine learning has a common assumption that training and test data are independent and identically distributed
that means samples are drawn from the same feature space and the same distribution.~\cite{daume2006}
When this assumption does not hold transfer learning and domain adaptation come into play.
Moreover, there is a biological inspiration
because humans seem to have natural ways to transfer knowledge from previous experience to new challenges.~\cite{torrey2010}

Pan and Yang define transfer learning as the ability of a system to recognise and apply knowledge and skill learned in previous tasks to novel tasks
and they introduce the notion of a domain and a task.
A domain consists of two components: a feature space and a marginal probability distribution.
Given a specific domain, a task consists of two components: a label space and an objective predictive function
which is not observed but learned from training data.
When we are given a transfer learning problem,
we have to identify a source domain and a source learning task,
a target domain and a target learning task.
Then, transfer learning aims to help improve the learning of the target predictive function in the target domain using knowledge in the source domain and the source task,
where the domains are different or the tasks are different.~\cite{pan2010}

Lastly, Torrey and Shavlik warn that both the source and target domains and tasks need to be sufficiently related
else negative transfer may occur.
Negative transfer is the situation in which the usage of source data degrades the performance.
On the other side, when the performance is improved,
we talk about a positive transfer.~\cite{torrey2010}

\section{Theory and Formalization of DA}

Domain adaptation is the scenario when the source and target domains have different marginal probability distributions
(for example two different telescopes observed the spectra)
while the tasks are the same
(we would like to classify them into the same classes).

\section{Neural Networks in DA}

\section{Previous Applications of DA in Astronomy}

As we have shown, domain adaptation is of great interest to astronomers
because of different instruments, measurements and observation distribution.
Therefore, we survey the current state of domain adaptation in astronomical applications in this section.

If we have a common set of observed stars in both archives,
then we can map them and learn a transfer function.
Ho et al.~\cite{ho2017} did exactly that
% TODO add APOGEE to acronyms
because they found a common set of 9952 spectra in both APOGEE and LAMOST archive.
Using that set, they trained the Cannon method~\cite{ness2015},
and they used the model to transfer some physical parameters from APOGEE to LAMOST.

In the case, when there is no common set Gupta et al. experimented with subspace alignment~\cite{fernando2014} and kernel mean matching~\cite{gretton2009} followed by active learning.~\cite{gupta2016}
In the case of subspace alignment, the negative transfer occurred while the kernel mean matching seems very promising in the task of supernova classification.
Then, Vilalta et al.~\cite{vilalta2018} extended the work of Gupta et al.~\cite{gupta2016}.
Vilalta et al. used a maximum a posteriori (MAP) approach to learn a prior on the model parameters from a spectroscopic source domain
and then use this prior distribution to learn a model in a photometric target domain.
Concretely, Vilalta et al. put a prior on the number of layers of a neural network
and then used active learning.
Richards et al.~\cite{richards2011} faced a similar situation, as Gupta et al.
Richards et al. introduce the problem as sample selection bias~\cite{shimodaira2000} or covariate shift~\cite{heckman1979}
when different distributions generate the source and target data.
That is precisely the problem we have defined as domain adaptation.
Richards et al. experimented with random forest in combination with three domain adaptation methods:
importance weighting~\cite{shimodaira2000}, co-training~\cite{blum1998} and active learning~\cite{settles2009}.
The result is not surprising from our point of view.
Active learning works best while importance weighting and co-training achieve negative transfer.

The term transfer learning has been recently used in the context of deep learning.
However, transfer learning in the context of deep learning means something more concrete than what we defined as transfer learning previously.
Transfer learning in the context of deep learning is the specific situation
when a pretrained deep neural network model is taken,
and its last layers are retrained with the target domain data.
Ackermann et al.~\cite{ackermann2018} employed with the transfer learning approach in the context of deep learning to detect galaxy merges.
Ackermann et al. took the Xception convolutional neural network~\cite{chollet2017}
and retrain its last layers with images of galaxy merger labelled in a citizen science Galaxy Zoo project.
This transfer learning approach allowed Ackermann et al. to lower the best error rate so far by 15\%.
