\documentclass[thesis=M,english]{FITthesis}[2012/10/20]

% graphics files inclusion
\usepackage{graphicx}
% subfigures
\usepackage{subfig}
% advanced maths
\usepackage{amsmath}
% additional math symbols
\usepackage{amssymb}
% directory tree visualisation
\usepackage{dirtree}

\department{Department of Applied Mathematics}
\title{Domain Adaptation with/using Neural Network Models between Source-One? and Source-Two? Astronomical Spectral Data}
\authorGN{Ondřej}
\authorFN{Podsztavek}
% author's name without academic degrees
\author{Ondřej Podsztavek}
\authorWithDegrees{Bc. Ondřej Podsztavek}
\supervisor{Petr Škoda}
\acknowledgements{TODO}
\abstractEN{TODO}
\abstractCS{TODO}
\placeForDeclarationOfAuthenticity{Prague}
\keywordsCS{TODO}
\keywordsEN{TODO}
% select according to the desired license (integer 1-6)
\declarationOfAuthenticityOption{2}
% TODO optional thesis URL
%\website{}

\begin{document}

\chapter{Introduction}

How to apply \textit{domain adaptation} using \textit{neural network models} to two different sources of \textit{astronomical spectral data}?

In this work, we would like to prove that astronomical spectroscopy can benefit from domain adapatation (DA).
Previous research has shown that DA can overcome the problem of analysis data from different distribution in one experiment.
In the first chapter~\ref{da_chapter}, we provide a summary of theory of DA, neural networks in DA and DA in astronomy.
Then, in chapter~\ref{data_chapter}, we describe two source of astronomical spectral data.
In experiments~\ref{exp_chapter}, we try to show the benefits of neural DA in astronomy.
Consider that our work is limited to astronomical spectroscopy and neural models.
Having said that, we will apply these methods\dots{}

\section{Problem Definition and Motivation}

\chapter{Domain Adaptation (DA)}
\label{da_chapter}

\section{DA in Context of Machine and Transfer Learning}

\section{Theory and Formalization of DA}

\section{Neural Networks in DA}

\section{Previous Applications of DA in Astronomy}

\chapter{Two Sources of Astronomical Spectral Data}
\label{data_chapter}

\section{Description of Source-One?}

\section{Description of Source-Two?}

\section{Comparison of Spectral Data of the Two Sources}

\chapter{Experiments: Applicatoin of Neural DA to Spectral Data}
\label{exp_chapter}

\section{Results without use of Neural DA}

\section{Benefits of Neural DA: Analysis and Visualisation of Results}

\chapter{Conclusion}

\section{Discussion of Performance}

\section{Future Plans}

\bibliographystyle{iso690}
\bibliography{references}

\appendix

\chapter{Acronyms}
\begin{description}
    \item[LAMOST] Large Sky Area Multi-Object Fiber Spectroscopis Telescope
\end{description}

\chapter{Contents of Enclosed CD}

\begin{figure}
\dirtree{%
    .1 README.md\DTcomment{the file with CD contents description}.
    .1 src\DTcomment{the directory of source codes}.
    .2 latex\DTcomment{the directory of \LaTeX{} source codes of the thesis}.
    .1 thesis.pdf\DTcomment{the thesis text in PDF format}.
}
\end{figure}

\end{document}
